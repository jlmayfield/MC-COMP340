\documentclass[10pt]{article}
\usepackage{amsmath}
\usepackage{setspace}


\setlength{\textheight}{9in} \setlength{\topmargin}{-.5in}
\setlength{\textwidth}{6.5in} \setlength{\oddsidemargin}{0in}
\setlength{\evensidemargin}{0in}

\title{COMP 340 - Paper Feedback Form}
\author{  }
\date{Spring 2014}

\begin{document}
\maketitle 
\thispagestyle{empty}

Paper Title: \underline{\hspace{4in}}
\vspace{.2in}

Author:\underline{\hspace{4in}}
\vspace{.2in}

Reviewer\underline{\hspace{4in}}
\vspace{.1in}

For all of the questions below a $1$ indicates that you strongly disagree, a $3$ is neither agree nor disagree and a $5$ means you strongly agree.
\begin{itemize}
\item I could understand the overall structure and organization of the paper form only a quick first read.
\[ 1 \quad 2 \quad 3 \quad 4 \quad 5\]
\item The introductory material provided sufficient context for the focus topic.
\[ 1 \quad 2 \quad 3 \quad 4 \quad 5\]
\item The technical focus material was clearly presented.
\[ 1 \quad 2 \quad 3 \quad 4 \quad 5\]
\item The conclusion provided some clear, specific insights as to the further usage/relevance of the focus topic.
\[ 1 \quad 2 \quad 3 \quad 4 \quad 5\]
\item Citations were present and properly used.
\[ 1 \quad 2 \quad 3 \quad 4 \quad 5\]
\item Overall, the writing style and mechanics contributed to my ability comprehend the material.
\[ 1 \quad 2 \quad 3 \quad 4 \quad 5\]
\item Overall, the paper's structure and organization contributed to my ability to comprehend the material.
\[ 1 \quad 2 \quad 3 \quad 4 \quad 5\]
\item I am interested in this subject matter.
\[ 1 \quad 2 \quad 3 \quad 4 \quad 5\]
\item I am comfortable with technical knowledge needed for this subject matter.
\[ 1 \quad 2 \quad 3 \quad 4 \quad 5\]
\newpage \thispagestyle{empty}

\item Write one clarifying question that you'd like to ask the author? A clarifying question should she more light on the basic material without going on beyond the core topic.
\vspace{1in}
\item Write one probing question that you'd like to ask the author?  A probing question should dig past the core topic and the immediate material presented in the paper.
\vspace{1in}
\item Additional Comments
\end{itemize}

\end{document}
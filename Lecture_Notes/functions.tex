\documentclass[]{tufte-handout}
\usepackage{amsmath, amssymb, amsthm}
\usepackage{braket}
\usepackage{setspace}
\usepackage{algorithm2e}

\newtheorem{theorem}{Theorem}
\newtheorem{definition}{Definition}
\newtheorem{remark}{Remark}

\title{Useful Equalities and Definitions}
\author{}
\date{}

\begin{document}
\maketitle

\begin{abstract}
A list of useful mathematical definitions for algorithm analysis and computer science in general. Much of this was copied with minor modification from:
\begin{itemize}
\item Cormen, Thomas H., et. al.. \textit{Introduction to Algorithms}. Second Edition. MIT Press. Cambridge, MA. 2001.
\end{itemize}

\end{abstract}


\section{Floors and Ceilings}

\begin{definition}
For real number $x$, the \textbf{floor} of $x$, $\lfloor x \rfloor$, is the greatest integer less than $x$.
\end{definition}

\begin{definition}
For real number $x$, the \textbf{ceiling} of $x$, $\lceil x \rceil$, is the least integer greater than $x$.
\end{definition}

For any real number $x$,
\begin{equation}
x-1 < \lfloor x \rfloor \leq x \leq \lceil x \rceil < x+1
\end{equation}

For any integer $n$,
\begin{equation}
\lceil n/2 \rceil + \lfloor n/2 \rfloor = n
\end{equation}

For any real $n \geq 0$ and integers $a,b > 0$:
\begin{eqnarray}
\lceil \lceil n/a \rceil / b\rceil = \lceil n/ab \rceil \\
\lfloor \lfloor n/a \rfloor / b \rfloor = \lfloor n/ab \rfloor \\
\lceil a/b \rceil \leq (a + (b-1))/b \\
\lfloor a/b \rfloor \geq (a - (b-1))/b 
\end{eqnarray}

\section{Polynomials}

\begin{definition}
Given integer $d>1$, a \textbf{Polynomial in $n$ of degree $d$} is a function $p(n)$,
\begin{equation}
p(n) = \sum\limits_{i=0}^{d}a_in^i
\end{equation}
where $a_0,a_1,\ldots,a_d$ are the \textbf{coefficients} and $a_d \neq 0$.
\end{definition}


\section{Exponentials}

For all real $a > 0,m,n$:
\begin{eqnarray}
a^0 = 1 \\
a^1 = a \\
a^{-1} = \frac{1}{a} \\
(a^m)^n = a^{mn} \\
(a^m)^n = (a^n)^m \\
a^ma^n = a^{m+n}
\end{eqnarray}


\section{Logarithms}

\begin{definition}
Where $b^y = x$, $\log_b x = y$.
\end{definition}

The following is only notation,
\[ \log_b^k n = (\log_b n)^k \]

For all $a>0$,$b>0$, $c>0$, $n$,
\begin{eqnarray}
a = b^{\log_b a} \\
\log_c (ab) = \log_c a + \log_c b \\
\log_b a^n = n \log_b a \\
\log_b a = \frac{\log_c a}{\log_c b} \\
\log_b \frac{1}{a} = -\log_b a \\
\log_b a = \frac{1}{\log_a b} \\
a^{\log_b c} = c^{\log_b a}
\end{eqnarray}

\section{Roots}

\begin{definition}
Where $r^n= x$, then the $n^{th}$ root of $x$ is $r$ and is denoted by $\sqrt[y]{x} = x^{\frac{1}{y}} = r$
\end{definition}

\begin{eqnarray}
\sqrt[n]{ab} = \sqrt[n]{a}\sqrt[n]{b} \\
\sqrt[n]{\frac{a}{b}} = \frac{\sqrt[n]{a}}{\sqrt[n]{b}} \\
\sqrt[n]{a^m} = (a^m)^{\frac{1}{n}} = a^{\frac{m}{n}}
\end{eqnarray}

\end{document}
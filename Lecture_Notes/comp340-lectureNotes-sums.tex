\documentclass[]{tufte-handout}
\usepackage{amsmath,amssymb,amsthm}
\usepackage{clrscode3e}

\newtheorem{theorem}{Theorem}
\newtheorem{lemma}{Lemma}
 
\title{COMP 340 - Summations}
\date{Spring 2014}


\begin{document}
\maketitle

When analyzing algorithms we often encounter summations of the following form. For integers $a$ and $b$ where $0 \leq a \leq b$.
\begin{equation}
\sum\limits_{i=a}^{b}f(i)
\end{equation}

Many variations on this are well documented:
\begin{equation}
\sum\limits_{i=0}^{n}i = \dfrac{n(n+1)}{2}
\end{equation}

\begin{equation}
\sum\limits_{i=0}^{n}i^2 = \dfrac{n(n+1)(2n+1)}{6}
\end{equation}

\begin{equation}
\sum\limits_{i=0}^{n}c^i = \dfrac{c^{n+1}-1}{c-1}
\end{equation}

The trick is to tease these closed-form formulas out of general summations rather than to attempt the full, open sum.

\section{Number of terms}

How many terms are there in the following sum?
\begin{equation*}
\sum\limits_{i=a}^{b}f(i)
\end{equation*}
If we write an expanded form, things become clear:
\begin{equation*}
\sum\limits_{i=a}^{b}f(i) = f(a) + f(a+1) + \ldots + f(b-1) + f(b)
\end{equation*}
There's one term per integer in the closed interval $[a,b]$.  
\begin{equation}
| [a,b] | = b-a+1
\end{equation}
It's always useful to have some sense of the \textsc{size} of the summation you're working with if for no other reason than to remind you that behind that compact notation is a long sequence of terms to be summed. 


\section{Summation Manipulation}

Some sums are really just multiplication in disguise.  If the thing being summed is independent of the summation variable, then really we're just looking at a fixed set of terms.
\begin{equation}
\sum\limits_{i=a}^{b}k = k(b-a+1)
\end{equation}
As always, a quick expansion gives insight as to what's going on.
\begin{equation*}
\begin{array}{rcl}
\sum\limits_{i=a}^{b}k &=& k + k + \ldots \\ \\
&=& 2k + \ldots
\end{array}
\end{equation*}

Let us now turn our attention to true sums. Our first manipulation rule we already know from algebra, we just have to recognize it in this new notation. We can always factor out common multiples from a sum. For some fixed value $s$
\begin{equation}
\sum\limits_{i=a}^{b}(s*f(i)) = s*\left(\sum\limits_{i=a}^{b}f(i)\right)
\end{equation}


The next rule comes from the commutativity of addition.  
\begin{equation}
\sum\limits_{i=a}^{b}(f(i)+g(i)) =\sum\limits_{i=a}^{b}f(i) + \sum\limits_{i=a}^{b}g(i)
\end{equation}
We can think of this as ``summations distribute over sums'' but really it's just a redistribution of terms, or a grouping of like terms.  If we write this whole thing out again, you should see the pattern.
\begin{equation*}
\begin{array}{rcl}
\sum\limits_{i=a}^{b}(f(i)+g(i)) &=& f(a)+g(a)+f(a+1)+g(a+1) \ldots \\ \\
 &=& f(a) + f(a+1) + g(a) + g(a+1) + \ldots
\end{array}
\end{equation*}
Because addition is commutative, we can rearrange to group the $f$ and $g$ terms together.

The final rule is maybe the most clever of the group. You might have noticed that our known sums are defined from 0 to $n$, but we're talking about sums from $a$ to $b$. No problem, just get the sum from 0 to $b$ then subtract off the sum from $0$ to $a-1$. 
\begin{equation}
\sum\limits_{i=a}^{b}f(i) = \sum\limits_{i=0}^{b} f(i) - \left( \sum\limits_{i=0}^{a-1} f(i)  \right)
\end{equation}
Notice the final product has both positive and negative terms for $f(i)$ for all the values of $i$ from 0 to $a-1$.  These all cancel and leave us with the terms from $a$ to $b$.  So, if all you can do is plug in formulas for known sums from $0$ to $n$, then this rule lets you reduce\sidenote{expand!} any $a$ to $b$ sum to $0$ to $n$ sums.

\section{Example}

The following examples uses all the rules and the known series  to get rid of all the summations and reduce a triple sum to basic algebra:
\begin{equation*}
\begin{array}{rcl}
\sum\limits_{i=1}^{n-1} \sum\limits_{j=i+1}^{n} \sum\limits_{k=1}^{j} 1 &=&
    \sum\limits_{i=1}^{n-1} \sum\limits_{j=i+1}^{n} j \\ \\
&=&
\sum\limits_{i=1}^{n-1} \left( \sum\limits_{j=0}^{n} j - \sum\limits_{j=0}^{i} j  \right) \\ \\
&=&
\sum\limits_{i=1}^{n-1} \left( \dfrac{n(n+1)}{2} - \dfrac{i(i+1)}{2}  \right) \\ \\
&=&
\sum\limits_{i=1}^{n-1} \dfrac{n(n+1)}{2} - \sum\limits_{i=1}^{n-1}  \dfrac{i(i+1)}{2}  \\ \\
&=&
(n-1)\dfrac{n(n+1)}{2}  - \left( \sum\limits_{i=1}^{n-1} \dfrac{i(i+1)}{2}  \right) \\ \\
&=& 
(n-1)\dfrac{n(n+1)}{2}  - \dfrac{1}{2}\left( \sum\limits_{i=1}^{n-1} i(i+1)  \right) \\ \\
&=&
(n-1)\dfrac{n(n+1)}{2}  - \dfrac{1}{2}\left( \sum\limits_{i=1}^{n-1} i^2 + \sum\limits_{i=1}^{n-1} i  \right) \\ \\
&=&
(n-1)\dfrac{n(n+1)}{2}  - \dfrac{1}{2}\left( \dfrac{n(n-1)(2n-1)}{6} + \dfrac{n(n-1)}{2} \right) 
\end{array}
\end{equation*}
The summations are now all gone. The one liberty I took above was with sums that start the summation variable at 1. For the types of sums we encountered, the $0^{th}$ term was always $0$, so starting from $1$ or $0$ is equivalent. We could have blindly used our expansion rule though. From here it's all basic algebra. Let's go ahead and simplify:
\begin{equation*}
\begin{array}{rcl}
(n-1)\dfrac{n(n+1)}{2}  - \dfrac{1}{2}\left( \dfrac{n(n-1)(2n-1)}{6} + \dfrac{n(n-1)}{2} \right) &=&
\dfrac{n(n-1)(n+1)}{2} - \dfrac{n(n-1)}{4}\left( \dfrac{(2n-1)}{3} + 1 \right) \\ \\
&=&
\dfrac{n(n-1)(n+1)}{2} - \dfrac{n(n-1)}{4}\left( \dfrac{(2n+2)}{3} \right) \\ \\
&=&
\dfrac{n(n-1)(n+1)}{2} - \dfrac{n(n-1)}{2}\left( \dfrac{(n+1)}{3} \right) \\ \\
&=&
\dfrac{n(n-1)(n+1)}{2} - \dfrac{n(n-1)(n+1)}{6} \\ \\ 
&=&
\dfrac{n(n-1)(n+1)}{3}  
\end{array}
\end{equation*}
Notice we took what is sure to be a lengthy series of terms and reduced it to just a few terms. That's pretty awesome; thank you closed form solutions! If you've done some integral calculus before, then you should be having some flashbacks involving $\int$ at this point as $\sum$ is the discrete counterpart to $\int$.  If you haven't taken calculus, no worries. 



\end{document}
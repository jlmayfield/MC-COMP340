\documentclass[]{tufte-handout}
\usepackage{amsmath,amssymb,amsthm}

 
\title{COMP 340 - Lecture Notes - 00 - Preface}
\date{Spring 2014}

\begin{document}
\maketitle

\begin{abstract}
Everything you need to know before reading Lecture notes for COMP340.
\end{abstract}

These lecture notes are intended to accompany the course texts\sidenote{

Skiena, Steven S. \textit{The Algorithm Design Manual}. Second Edition. Springer. London. 2012. ISBN: 978-1-84800-069-8
\vspace{.15in} \newline

Blelloch, Guy E. and Maggs, Bruce M. ``Parallel Algorithms.'' \textit{Computer Science Handbook.} Second Edition. Tucker, Allen B. (Ed.). Boca Raton, FL.: Chapman \& Hall/CRC, 2004. 25--1 - 25--44.
}
and not replace them.  They're generally written from the perspective that you've read the chapter and are looking for more insight.  So, where the main text will often work through an example, try things that fail, and finally arrive at a solution.  These notes will approach the examples with perfect hindsight\sidenote{which we never have in practice}.  We'll analyze the process, the mistakes, and the result to be certain the big picture was not missed.  So, they're meant to aid in reflection on reading done.\sidenote{So go do the reading!} 

Finally, it should be noted that these notes are drafts and so typos and errors should be expected. Reader be warned.

\end{document}

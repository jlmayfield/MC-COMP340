\documentclass[10pt]{article}
\usepackage{amsmath}
\usepackage{setspace}
\usepackage{hyperref}


\setlength{\textheight}{9in} \setlength{\topmargin}{-.5in}
\setlength{\textwidth}{6.5in} \setlength{\oddsidemargin}{0in}
\setlength{\evensidemargin}{0in}

\title{Syllabus - COMP 340 - Analysis of Algorithms}
\author{ James \textit{Logan} Mayfield }
\date{ Spring 2014}

\begin{document}
\maketitle

\section{Logistics}
\begin{itemize}
\item \textbf{Where: } Center for Science and Business, Room 303
\item \textbf{When: } MTWF, 9:-9:50am
\item \textbf{Instructor :} James \textit{Logan} Mayfield
\begin{itemize}
\item \textit{Office: } Center for Science and Business, Room 344
\item \textit{Phone: } 309-457-2200
\item \textit{Email: } lmayfield \textit{at} monmouthcollege \textit{dot} edu
\item \textit{Office Hours: } By Appointment*
\end{itemize}
\item \textbf{Credits: } 1 course credit
\end{itemize}
\emph{Note: This Syllabus is subject to change based on specific class needs. Deviations from the syllabus will be discussed in class.}


\section{Text}


Skiena, Steven S. \textit{The Algorithm Design Manual}. Second Edition. Springer. London. 2012. 
\newline ISBN: 978-1-84800-069-8
\vspace{.15in} \newline

The following survey article will be used as the basis for our studies of parallel algorithms.  It is freely available online; a printed copy will be made available to you.
\vspace{.25in}
\newline 


Blelloch, Guy E. and Maggs, Bruce M. ``Parallel Algorithms.'' \textit{Computer Science Handbook.} Second Edition. Tucker, Allen B. (Ed.). Boca Raton, FL.: Chapman \& Hall/CRC, 2004. 25--1 - 25--44.

\section{Description and Content}

Algorithms are one of the most crucial areas of study in Computer Science.  In this course, students will learn the basic tools of algorithm design and analysis through the study of some of the most well known and important algorithms.  By the end of the semester, students will have developed not only a firm grounding in the analysis and design of algorithms, but working knowledge of the algorithms that have made computing what it is today.

While it certainly is possible to study algorithms in the absence of programming, concrete implementations provide a tangible means of playing with the course material.  As a part of the class, students will implement, \textit{in the language of their choice}, present and demonstrate the algorithms from the text. Regular presentations of code will provide a backdrop for discussions of the relationships between programming, algorithms and the science of computing. 

\subsection{Content}

This course will emphasize the first eight chapters of the text interleaved with selections from Belloch and Maggs:
\begin{itemize}
\item Basic Design and Analysis [Chapters 1-2]
\item Parallel Design and Analysis [ 25.1-2]
\item Standard Data Structures [Chapter 3]
\item Sorting and Searching [Chapter 4]
\item Basic Parallelism [25.3-4,6]
\item Graph Traversal [Chapter 5]
\item Weighed Graph Algorithms [Chapter 6]
\item Parallelism for Graphs [25.5]
\item Combinatorial Search \& Heuristics [Chapter 7]
\item Dynamic Programming [Chapter 8]
\end{itemize}
When possible, we'll sprinkle in some basics of parallel algorithms as well. Time permitting, we'll examine chapter 9 and some basic issues in NP-Completeness. 



\section{Expectations and Policies}
The expectations for students in this course are not at all unreasonable.  To avoid any confusion, they are listed here.  These aren't necessarily rules but rather guidelines for how you should conduct yourself in this class.  Strict rules will result from these expectations and will be covered later.
\begin{itemize}
\item Be respectful of others.  Don't create unnecessary distractions.  Turn cell phones off, on silent or leave them in the dorm.  Class time is not the time for checking email, surfing the web and IMing.  \textit{Come to class ready and interested in learning and if you're not, don't behave in such a way that prevents others from doing so.}
\item You're in college.  College is meant to provide an education.  Therefore, you are, for all intents and purposes, a \textit{professional student}.  Your work should reflect a solid level of professionalism and be neat and orderly.  Take the extra time to make it presentable.  Crumpled papers with various liquid stains on them are not presentable.  Think of the instructor as your boss and that the quality of your paycheck depends on the quality of the work.  \textit{You don't have to always love the work you do, but you should always do it to the best of your capabilities.}
\item Attending class is not by itself sufficient for learning the material.  You're expected to read the sections of the text as they are covered in class.  You are encouraged to go beyond the material.  Make use of available resources such as tutors and the high availability of your instructor.  \textit{Don't expect to get an A just by showing up and doing the least amount of work that you can.}
\end{itemize}

There are several strict policies that result from these expectations.  In the case of these items, they are rules and you are responsible for understanding and abiding by them.
\begin{itemize}
\item \textit{Late Assignments: }In general, late assignments will \textit{not} be accepted.  If you feel you have a justified reason for the assignment being late you may set up an appointment to meet with the instructor and plead your case.  Situations beyond your control are understandable and exceptions can and will be made.
\item \textit{Attendance: }You're an adult, you can choose to not come to class.  If you do miss/skip class \textit{you are still responsible for everything covered on that day}.  If you have no valid reason for missing class, do not expect the instructor to spend the time to re-present the class to you individually.
\item \textit{Participation: }  Cellphone usage in class is not allowed, this includes text messages.  Turn off the ringers or leave them at home.  Computer usage is limited to activities in support of the course.  This does not include IMs, Facebook, checking email, general web surfing, poker, fantasy sports leagues, forum trolling, mine sweeper, etc.  This behavior is rude and can be a real distraction to others.  Repeated failure to abide by this policy will have a negative effect on your grade.  
\item \textit{Quality of Work:} There are several minimal requirements that your assignments must meet.
\begin{itemize}
\item \textit{Staples - } Assignments that take up more than one page must be stapled.  Unstapled assignments will either be returned to you to be stabled ASAP or points will be deducted.  
\item \textit{Neatness - }  Make every attempt to make your work neat and orderly:  label problems, avoid excessive scratching out of mistakes (use pencil if you are prone to errors) and if you use spiral bound paper tear off the edges. 
\item \textit{Show Work - } Rarely are answers alone sufficient for full credit.  Show your work whenever prudent.  If you're unsure if work is needed, \textit{ask!}
\end{itemize}
\end{itemize}

\subsection{Collaboration}

In general, you are encouraged to make use of the resources available to you.  This means it is OK to seek help from a friend, tutor, instructor, internet, etc.  However, \textit{copying of answers and any act worthy of the label of ``cheating'' is never permissible!}  It is understandable that when you work with a partner or a group that the resultant product is often extremely similar.  This is acceptable but be prepared to be asked to defend your collaborations to the instructor.  \textit{You should always be able to reproduce an answer on your own, and if you cannot you likely \textbf{do not really known the material.}} 
\begin{itemize}
\item When assignments are meant to be done in groups, you will be directed to turn in one set of solutions per group.
\item All other assignments should represent your own work and effort.
\end{itemize}
All of the Monmouth College rules on academic dishonesty apply.  If you violate the rules be prepared to face the consequences of your actions.  

\section{Grades}

This courses adheres to the Mathematics and Computer Science grade scale.  Assignments and final grades will not be curved except when deemed necessary by the instruction.  Percentage grades translate to letter grades as follows:
\newline
\begin{small}
\begin{tabular}{lc}
94-100 & A \\
90-93 & A- \\
88-89 & B+ \\
82-87 & B \\
79-81 & B- \\
76-78 & C+ \\
70-75 & C \\
67-69 & C- \\
64-66 & D+ \\
58-63 & D \\
55-57 & D- \\
0-54 & F 
\end{tabular}
\end{small}
\newline
You are always welcome to challenge a grade that you feel is unfair or calculated incorrectly.  Mistakes made in your favor will never be corrected to lower your grade.  Mistakes made not in your favor will be corrected.  \textit{Basically, after the initial grading your score can only go up as the result of a challenge.}

\subsection{Grade Weights}
Your final grade is based on a weighted average of particular assignment categories.  You should be able to estimate your current grade based on your scores and these weights.  You may always visit the instructor \textit{outside of class time} to discuss your current standing.  
\begin{itemize}
\item Homework 15\%
\item Paper + Presentation 20\%
\item Final 15\%
\item Midterm 10\%
\item Quizzes 30\%
\item Participation 10\%
% add more if needed
\end{itemize} 

\subsection{Workload}
% number of/details on midterms, finals, project, homeworks, quizes, etc

\begin{itemize}
\item 5 Homework Assignments
\item 1 Paper with Presentation
\item 1 Final
\item 1 Midterm
\item 5 Quizzes
\end{itemize}

\subsubsection{Calendar}

The following calendar should give you a feel for how work is distributed throughout the semester.  \textit{This calendar is subject to change based on the circumstances of the course.}

\begin{center}
\begin{tabular}{|c|c|r|}
\hline 
Week & Dates & Assignments \\
\hline
1 & 1/13-1/17 &  \\
\hline
2 & 1/20-1/24 & Homework 1.\\
\hline
3 & 1/27 - 1/30 &  Quiz 1.\\
\hline
4 & 2/3 - 2/7 &  Homework 2. \\
\hline
5 & 2/10 - 2/14 &  Quiz 2.\\
\hline
6 & 2/17 - 2/21 &  Homework 3.\\
\hline
7 & 2/24 - 2/28 &  \\
\hline
8 & 3/3 - 3/7 &  MIDTERM.\\
\hline 
SPRING BREAK & 3/10 - 3/14& \\
\hline
9 & 3/17 - 3/21 & Homework 4.\\
\hline
10 & 3/24 - 3/28 & Quiz 3.\\
\hline
11 & 3/31 - 4/4 &  Homework 5.\\
\hline
12 & 4/7 - 4/11 &   Quiz 4.\\
\hline
13 & 4/14 - 4/18 &  Papers. EASTER BREAK (Friday).\\
\hline
14 & 4/21 - 4/25 & EASTER BREAK (Monday).  Presentations. \\
\hline
15 & 4/28 - 5/2 &  Quiz 5.\\ 
\hline
16 & 5/5 - 5/7 & \\
\hline
Final's Week & 5/14 (8-11am) & Final Exam. \\ 
\hline
\end{tabular}
\end{center}

\end{document}
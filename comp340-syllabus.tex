\documentclass[10pt]{article}
\usepackage{amsmath}
\usepackage{setspace}
\usepackage{hyperref}
\usepackage{booktabs}
\usepackage{longtable}

\setlength{\textheight}{9in} \setlength{\topmargin}{-.5in}
\setlength{\textwidth}{6.5in} \setlength{\oddsidemargin}{0in}
\setlength{\evensidemargin}{0in}

\title{Syllabus - COMP 340 - Analysis of Algorithms}
\author{  }
\date{ Spring 2022}

\begin{document}
\maketitle

\section{Logistics}
\begin{itemize}
\item \textbf{Where: } Center for Science and Business, Room 303
\item \textbf{When: } MWF, 10-10:50am
\item \textbf{Instructor :} Logan Mayfield
\begin{itemize}
\item \textit{Office: } Center for Science and Business, Room 344
\item \textit{Phone: } 309-457-2200
\item \textit{Website: } \url{http://jlmayfield.github.io/}
\item \textit{Email: } lmayfield \textit{at} monmouthcollege \textit{dot} edu
\item \textit{Office Hours: } By Appointment
\end{itemize}
\item \textbf{Website: } \url{http://jlmayfield.github.io/teaching/COMP340/}
\item \textbf{Credits: } 1 course credit
\end{itemize}
\emph{Note: This Syllabus is subject to change based on specific class needs. Deviations from the syllabus will be discussed in class.}

\section{Description \& Content}


Algorithms are one of the most crucial areas of study in Computer Science.  In this course, students will learn the basic tools of algorithm design and analysis through the study of some of the most well known and important algorithms.  By the end of the semester, students will have developed not only a firm grounding in the analysis and design of algorithms, but working knowledge of the algorithms that have made computing what it is today.

While it certainly is possible to study algorithms in the absence of programming, concrete implementations provide a tangible means of playing with the course material.  As a part of the class, students will design, implement, and present algorithmic challenges from the text. Regular presentations of code will provide a backdrop for discussions of the relationships between programming, algorithms and the science of computing.

\section{Texts}

Skiena, Steven S. \textit{The Algorithm Design Manual}. Third Edition. Springer. London. 2020.
\newline ISBN: 978-3-030-54256-6.
\vspace{.15in} \newline

Skiena, Steven S. ``The Algorithm Design Manual, 3rd Edition''. \textit{Steven Skiena}. 4 Jan, 2022. \textit{algorist.com}.
\vspace{.15in} \newline


\subsection{Topics}

This course will cover the bulk of chapters 1--5, 7--10, and 13. The overall flow of the course is as follows
\begin{itemize}
\item Why? [Chapter 13]
\item Basic Design and Analysis [Chapters 1-2]
\item Building Blocks: Data Structures \& Sorting [Chapters 3--4]
\item Divide and Conquer [Chapter 5]
\item Graphs [Chapters 7--8]
\item Combinatorial Search [Chapter 9]
\item Dynamic Programming [Chapter 10]
\end{itemize}

Time and interest permitting, we'll explore the remaining topics from the text:
\begin{itemize}
  \item La Dura Dura [Chapters 11--12]
  \item That's so Random [Chapter 6]
\end{itemize}

\section{Workload}

The course workload is as follows:
\begin{center}
  \begin{tabular}{ll}
    \underline{Category} & \underline{Number of Assignments} \\
     Homework: Chapter Exercises & 6--7 \\
     Problems: Leetcode \& Interview Problems & 6--7 \\
     Exams & 6--7 \\
     Project & 1 \\
   \end{tabular}
 \end{center}

 \subsection*{Exams}

 All exams are weighted equally. There is no midterm or final exam in the sense that the exams are worth more than other exams or that they will necessarily take longer than other exams.  Exams will cover material covered since the previous exam. Unless stated otherwise, assume that exams will be pencil and paper and that computers will not be available during the exam period.

\subsection*{Homework}

Homework will consist of exercises from the end of the chapters. Expect them to be assigned in bulk, before the material is covered and due right as we conclude the relevant material. The intent is for you to work ahead and work as we go rather than wait until we're done to start. Working in groups is encouraged but work should still be done honestly.

\subsection*{Program}

Program assignments consist of either Leet Code or Interview problems from the end of the chapters. Students can expect to choose one or two options. Some, but not all, problems will be presented to the class as part of the assignment.

\subsection*{Project}

At the end of the semester students will complete one of the chapter implementation or design challenges. They will then present their solution along with relevant algorithmic problems from section II of the text. The overall emphasis of this project is connecting concrete algorithmic work to general algorithmic problems and not simply to complete the challenge.

\subsection{Course Engagement Expectations}

 The weekly workload for this course will vary by student but on average should be about 13 hours per week.  The follow tables provides a rough estimate of the distribution of this time over different course components. Study time and Project time is amortized for the whole semester, but the work will, as usual, come in spurts.
 \begin{center}
 \begin{tabular}{ll}
 \underline{Assignment Type} & \underline{Time/week} \\
 Lectures       & 3 hours/week \\
 Homework \& Programs & 3 hours/week \\
 Exam Study Time    & 1 hours/week \\
 Project          & 1 hours/week \\
 Reading &  2 hours/week \\
 \bottomrule
  & 10 hours/week
 \end{tabular}
 \end{center}


\section{Grades}

This course uses a standard grading scale where percentage grades translate to letter grades as follows:

\begin{center}
\begin{small}
\begin{tabular}{lcl}
\underline{Score} & & \underline{Grade} \\
94--100 & & A \\
90--93 & & A- \\
88--89 & & B+ \\
82--87 & & B \\
80--81 & & B- \\
78--79 & & C+ \\
72--77 & & C \\
70--71 & & C- \\
68--69 & & D+ \\
62--67 & & D \\
60--61 & & D- \\
0--59 & & F
\end{tabular}
\end{small}
\end{center}

Students are always welcome to challenge a grade that they feel is unfair or calculated incorrectly.  Mistakes made in the student's favor will never be corrected to lower the grade.  Mistakes not in the student's favor will be corrected.  \textit{Basically, after the initial grading, a score can only go up as the result of a challenge.}

\subsection{Grade Weights}

The final grade is based on a weighted average of particular assignment categories.  Students should be able to estimate your current grade based on your scores and these weights, but you may always visit the instructor \textit{outside of class time} to discuss your current standing and check on some or all of the current course grade.

\begin{center}
  \begin{tabular}{ll}
  \underline{Category} & \underline{Weight} \\
    % Assuming 6 Hwk+Prog+Exam Combos
    Exams & 42\% \\ %7 each
    Homework & 18\% \\ %3 each
    Programs & 24\% \\ %4 each
    Project & 6\% \\ %6 each
    Participation & 10\%
  \end{tabular}
\end{center}

\subsection{Participation, Attendance, \& Late Assignments}

%This course will make almost daily use of Socrative for in-class question and answer sessions. Questions will cover portions of the text that were assigned as reading and will range from simple checks to see if the reading was done to more challenging questions that follow from a close examination of the reading.  For the most part, the only requirement is to provide an answer to every question and participate in the resultant discussions.  On occasion, questions will be evaluated for their correctness and performance on these questions will also factor into the course participation grade.  Students who do the reading and start the homework as soon as possible will have very little to worry about.

%While there is no strict attendance policy, the course participation grade is based in large part on engagement with socrative. Absent students cannot participate in socrative sessions.  Students should avoid unexcused absences, as defined in the college-wide absence policy. Whenever possible, let the instructor know of the absence before it occurs. When unexcused absences do occur, it is the student's responsibility to make up for the lost class time and to seek the permission of the instructor to hand-in or complete assignments that are late due to an unexcused absence.

As a rule, assignments are due at the specified time and no late assignments will be accepted unless an extension was requested prior to the due date. There are, of course, exceptions to this rule and students needing extra time can always contact the instructor for an extension. Do not just give up and eat a zero for the assignment. Ever. There is no penalty in asking for an extension nor is there a limit on extensions.  That being said, there is no guarantee an extension will be given without legitimate need.

This course is designed around the assumption that students \textit{engage in new ideas before they're covered in class meetings}.  This means doing assigned reading, taking a stab at homework problems, and as a result coming to class and lab with some understand about a new idea or, just as likely, with a host of questions about something encountered in the reading and homework. Not attending class, skipping lab, and putting off work to the point that an extension is needed are signs that a student isn't holding up their end of the bargain and is not prepared to participate in class.

\subsection{Academic Honesty}

From the Monmouth College Academic Honesty Policy:
\begin{quote}
  ``We view academic dishonesty as a threat to the integrity and intellectual mission of our institution. Any breach of the academic honesty policy - either intentionally or unintentionally - will be taken seriously and may result not only in failure in the course, but in suspension or expulsion from the college. It is each student’s responsibility to read, understand and comply with the general academic honesty policy at Monmouth College, as defined here in the Scots Guide, and to the specific guidelines for each course, as elaborated on the professor’s syllabus.''

  ``The following areas are examples of violations of the academic honesty policy:
  \begin{enumerate}
  \item Cheating on tests, labs, etc;
  \item Plagiarism, i.e., using the words, ideas, writing, or work of another without giving appropriate credit;
  \item Improper collaboration between students, i.e., not doing one’s own work on outside assignments specified as group projects by the instructor;
  \item Submitting work previously submitted in another course, without previous authorization by the instructor.''
  \end{enumerate}

  ``Please note that this list is not intended to be exhaustive.''
\end{quote}

The complete Monmouth College Academic Honesty Policy can be found on the College web page by clicking on ``Student Life'' then on ``Scot’s Guide'' in the navigation bar to the left, then ``Academic Regulations'' in the navigation bar at the left.  Or you can visit the web page directly by typing in this URL: \url{https://ou.monmouthcollege.edu/life/residence-life/scots-guide/academic-regulations.aspx}

In this course, any violation of the academic honesty policy will have varying consequences depending on the severity of the infraction as judged by the instructor. Minimally, a violation will result in an``F'' or 0 points on the assignment in question. Additionally, the student’s course grade may be lowered by one letter grade. In severe cases, the student will be assigned a course grade of ``F'' and dismissed from the class. All cases of academic dishonesty will be reported to the Associate Dean who may decide to recommend further action to the Admissions and Academic Status Committee, including suspension or dismissal. It is assumed that students will educate themselves regarding what is considered to be academic dishonesty, so excuses or claims of ignorance will not mitigate the consequences of any violations.

\section{Accessibility}

Student Success \& Accessibility Services offers FREE resources to assist Monmouth College students with their academic success. Programs include Supplemental Instruction for select classes, Drop-In and appointment tutoring, and individual Academic Coaching. Our office is here to help all students excel academically, since all students can work toward better grades, practice stronger study skills, and manage their time better.

If you have a disability or had academic accommodations in high school or another college, you may be eligible for academic accommodations at Monmouth College under the Americans with Disabilities Act (ADA). Monmouth College is committed to equal educational access. To discuss any of the services offered, please call or meet with Robert Crawley, Interim Director of Student Success \& Accessibility Services.  SSAS is located in the new ACE space on the first floor of the Hewes Library, opposite Einstein’s Bros Bagels. They can be reached at 309-457-2257 or via email at: ssas@monmouthcollege.edu.

\section{Calendar}

The overall flow of the course and the work you'll be doing is as follows:
\begin{center}
\begin{tabular}{lll}
\underline{Section} & \underline{Chapters} & \underline{Assignments} \\
Why? &  13 & \\
Reasoning &  1--2 & Homework 1. Program 1. Exam 1. \\
Building Blocks &  3--4 & Homework 2. Program 2. Exam 2. \\
Divide \& Conquer & 5 & Homework 3. Program 3. Exam 3. \\
Graphs & 7--8 & Homework 4. Program 4. Exam 4. \\
All the Things $\ldots$ ish & 9 & Homework 5. Program 5. Exam 5. \\
Dynamic Programming & 10 & Homework 6. Program 6. Exam 6. \\
Project Time! & 1--13 & Project 1. \\
\textit{Time Permitting} & & \\
La Dura Dura & 11--12 &   \\
Rando & 6 &
\end{tabular}
\end{center}

This workflow will roughly map to this calendar. \textit{This calendar is subject to change based on the circumstances of the course.} Precise dates and other day-to-day details can be found on the course website.

\begin{center}
\begin{tabular}{lllll}
\underline{Week} & \underline{Dates} & \underline{Notes} & \underline{Assignments Due} & \underline{Chapter(s)}\\
1 & 1/11 --- 1/14 & & & 13,1   \\
2 & 1/17 --- 1/21 & & & 1,2 \\
3 & 1/24 --- 1/28 & & Homework 1. Program 1. & 2,3 \\
4 & 1/31 --- 2/4 & & Exam 1. & 3,4 \\
5 & 2/7 --- 2/11 & &  Homework 2. Program 2. & 4 \\
6 & 2/14 --- 2/18 & & Exam 2 & 5 \\
7 & 2/21 --- 2/25 & & Homework 3. Program 3. & 5,7 \\
8 & 2/28 --- 3/4 &  SPRING BREAK (F) & Exam 3. & 7  \\
& 3/7 --- 3/11 &  SPRING BREAK & \\
9 & 3/14 --- 3/18 & & & 8 \\
10 & 3/21 --- 3/25 & & Homework 4. Program 4. & 9 \\
11 & 3/28 --- 4/1 & & Exam 4. Homework 5. Program 5. & 9,10\\
12 & 4/4 --- 4/8 &  & Exam 5. &  10\\
13 & 4/11 --- 4/15 & EASTER (F) & Homework 6. Program 6. &  6,11--12\\
14 & 4/18 --- 4/22 & EASTER (M). & Exam 6. & 6,11--12\\
15 & 4/25 --- 4/29 & SCHOLAR'S DAY (Tu). & Homework 7. Program 7. Project &  6,11--12\\
16 & 5/2 --- 5/6 & READING DAY (Th) & Exam 7 (Sat. 5/7, 8--11 am) &
\end{tabular}
\end{center}

\end{document}
